\documentclass{article}

\usepackage[utf8x]{inputenc}
\usepackage[T1]{fontenc}
\usepackage[francais]{babel}
\usepackage{xcolor}
\usepackage{listings}
\usepackage{mathptmx}
\usepackage{anyfontsize}
\usepackage{t1enc}
\usepackage[top=2cm, bottom=2cm, left=2cm, right=2cm]{geometry}
\usepackage{titlesec}
\usepackage{titling}
\usepackage{graphicx}
\usepackage{pgfplots}
\usepackage[colorlinks = true,
            linkcolor = black,
            urlcolor  = black,
            citecolor = black,
            anchorcolor = black]{hyperref}

\newcommand{\changeurlcolor}[1]{\hypersetup{urlcolor=#1}}

\renewcommand\maketitlehooka{\null\mbox{}\vfill}
\renewcommand\maketitlehookd{\vfill\null}

\definecolor{codegreen}{rgb}{0,0.6,0}
\definecolor{codegray}{rgb}{0.5,0.5,0.5}
\definecolor{codepurple}{rgb}{0.58,0,0.82}
\definecolor{backcolour}{rgb}{0.95,0.95,0.92}
\definecolor{codekeywords}{rgb}{0.1,0.53,0.92}

\lstdefinestyle{c++}{
    backgroundcolor=\color{backcolour},   
    commentstyle=\color{codegreen},
    keywordstyle=\color{codekeywords},
    numberstyle=\tiny\color{codegray},
    stringstyle=\color{codepurple},
    basicstyle=\ttfamily\footnotesize,
    breakatwhitespace=false,         
    breaklines=true,                 
    captionpos=b,                    
    keepspaces=true,                 
    numbers=left,                    
    numbersep=5pt,                  
    showspaces=false,                
    showstringspaces=false,
    showtabs=false,                  
    tabsize=2,
    texcl=false,
    inputencoding=utf8,
    extendedchars=true,
    literate=
  {á}{{\'a}}1 {é}{{\'e}}1 {í}{{\'i}}1 {ó}{{\'o}}1 {ú}{{\'u}}1
  {Á}{{\'A}}1 {É}{{\'E}}1 {Í}{{\'I}}1 {Ó}{{\'O}}1 {Ú}{{\'U}}1
  {à}{{\`a}}1 {è}{{\`e}}1 {ì}{{\`i}}1 {ò}{{\`o}}1 {ù}{{\`u}}1
  {À}{{\`A}}1 {È}{{\'E}}1 {Ì}{{\`I}}1 {Ò}{{\`O}}1 {Ù}{{\`U}}1
  {ä}{{\"a}}1 {ë}{{\"e}}1 {ï}{{\"i}}1 {ö}{{\"o}}1 {ü}{{\"u}}1
  {Ä}{{\"A}}1 {Ë}{{\"E}}1 {Ï}{{\"I}}1 {Ö}{{\"O}}1 {Ü}{{\"U}}1
  {â}{{\^a}}1 {ê}{{\^e}}1 {î}{{\^i}}1 {ô}{{\^o}}1 {û}{{\^u}}1
  {Â}{{\^A}}1 {Ê}{{\^E}}1 {Î}{{\^I}}1 {Ô}{{\^O}}1 {Û}{{\^U}}1
  {œ}{{\oe}}1 {Œ}{{\OE}}1 {æ}{{\ae}}1 {Æ}{{\AE}}1 {ß}{{\ss}}1
  {ç}{{\c c}}1 {Ç}{{\c C}}1 {ø}{{\o}}1 {å}{{\r a}}1 {Å}{{\r A}}1
  {€}{{\EUR}}1 {£}{{\pounds}}1,
}
\lstset{style=c++}


\title{Simulation TP5\\Système Multi-Agents}
\author{Arquillière Mathieu - Zangla Jérémy}
\date{\today}

\begin{document}

\begin{titlepage}
  \maketitle
\end{titlepage}

\tableofcontents
\newpage
\listoffigures
\newpage

\section{Introduction}
L'objectif ici est de réaliser un système multi-agents simple, afin d'en
comprendre et d'expérimenter les difficultés et les intérêts.

\section{Modélisation}
\subsection{Agents}
Notre système se composera de 2 types d'agents:
\begin{itemize}
  \item les \emph{récolteurs}
  \item les \emph{mangeurs}
\end{itemize}

\subsubsection{Récolteurs}
L'objectif de cet agent est récolter des ressources et de les ramener à sa base. Chaque
agent de ce type a été créé par une \emph{case particulière} de l'environnment qu'on
appelera sa \emph{base}. Un \emph{récolteur} bouge aléatoirement dans l'espace
en 2 dimensions. Lorqu'il trouve une case avec des ressources, il la prend et se déplace
alors vers leur base (ce qui implique que ces agents savent à tous moments l'endroit de
leur base). Une fois dessus, il dépose ses ressources et se remet en recherche.

Les \emph{récolteurs} naissent grâce à une base. Lorsqu'une base possède assez de ressources,
elle les consomme et créé un nouvel agent \emph{récolteur}.

\begin{figure}[!ht]
  \centering
  \caption{Diagramme état-transition de l'agent \emph{récolteur}}
  \includegraphics[scale=0.70]{img/etat-transition_recolteur.png}
\end{figure}

\subsection{Mangeurs}
L'objectif de cet agent est de manger des agents \emph{récolteurs} afin de survivre.
Il se déplace aléatoirement jusqu'à ce qu'il trouve dans un voisinage de Moore d'ordre 3
un agent \emph{récolteur}. Dès lors, il se déplace vers celui-ci (ses déplacements se
font dans un voisinage de Moore d'ordre 2). Si il atteint un \emph{récolteur} (être
sur la même case de l'esapce 2D) alors il le détruit.

Le \emph{mangeur} possède une "barre de vie" qui diminue à chaque nouvel état du système.
Manger un \emph{récolteur} permet de regagner de la vie. Si il en mange un alors que sa
vie était assez haute (exemple: > 90\%) alors il créé un nouvel agent \emph{mangeur}.

\begin{figure}[!ht]
  \centering
  \caption{Diagramme état-transition de l'agent \emph{mangeur}}
  \includegraphics[scale=0.75]{img/etat-transition_mangeur.png}
\end{figure}

\subsection{Environnement}
L'environnement sera une matrice (20x20 à la base). Une case de cette matrice peut:
\begin{itemize}
  \item être vide
  \item contenir un agent
  \item contenir une ressource
  \item contenir une base
\end{itemize}

\subsubsection{}
\begin{itemize}
  \item les ressources
  \item les bases
\end{itemize}

Une ressource a une position dans l'espace 2D et a un type : faible, moyen ou fort et
correspond à une quantité pour les bases. Tous les certains temps du système, un certain
nombre de ressources apparaissent aléatoirement dans l'espace.

Une base intéragit avec des agents \emph{récolteurs}. Elle possède une position fixe
dans l'espace 2D et s'occupe de créer des \emph{récolteurs}. En effet ceux-ci rapporte
à la base des ressources qui, au dessus d'un certain seuil, sont consommées afin de
générer un nouvel agent \emph{récolteur}.

\end{document}