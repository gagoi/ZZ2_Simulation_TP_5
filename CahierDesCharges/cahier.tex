\documentclass{article}

\usepackage[utf8x]{inputenc}
\usepackage[T1]{fontenc}
\usepackage[francais]{babel}
\usepackage{xcolor}
\usepackage{listings}
\usepackage{mathptmx}
\usepackage{anyfontsize}
\usepackage{t1enc}
\usepackage[top=2cm, bottom=2cm, left=2cm, right=2cm]{geometry}
\usepackage{titlesec}
\usepackage{titling}
\usepackage{graphicx}
\usepackage{pgfplots}
\usepackage[colorlinks = true,
            linkcolor = black,
            urlcolor  = black,
            citecolor = black,
            anchorcolor = black]{hyperref}

\newcommand{\changeurlcolor}[1]{\hypersetup{urlcolor=#1}}

\renewcommand\maketitlehooka{\null\mbox{}\vfill}
\renewcommand\maketitlehookd{\vfill\null}

\definecolor{codegreen}{rgb}{0,0.6,0}
\definecolor{codegray}{rgb}{0.5,0.5,0.5}
\definecolor{codepurple}{rgb}{0.58,0,0.82}
\definecolor{backcolour}{rgb}{0.95,0.95,0.92}
\definecolor{codekeywords}{rgb}{0.1,0.53,0.92}

\lstdefinestyle{c++}{
    backgroundcolor=\color{backcolour},   
    commentstyle=\color{codegreen},
    keywordstyle=\color{codekeywords},
    numberstyle=\tiny\color{codegray},
    stringstyle=\color{codepurple},
    basicstyle=\ttfamily\footnotesize,
    breakatwhitespace=false,         
    breaklines=true,                 
    captionpos=b,                    
    keepspaces=true,                 
    numbers=left,                    
    numbersep=5pt,                  
    showspaces=false,                
    showstringspaces=false,
    showtabs=false,                  
    tabsize=2,
    texcl=false,
    inputencoding=utf8,
    extendedchars=true,
    literate=
  {á}{{\'a}}1 {é}{{\'e}}1 {í}{{\'i}}1 {ó}{{\'o}}1 {ú}{{\'u}}1
  {Á}{{\'A}}1 {É}{{\'E}}1 {Í}{{\'I}}1 {Ó}{{\'O}}1 {Ú}{{\'U}}1
  {à}{{\`a}}1 {è}{{\`e}}1 {ì}{{\`i}}1 {ò}{{\`o}}1 {ù}{{\`u}}1
  {À}{{\`A}}1 {È}{{\'E}}1 {Ì}{{\`I}}1 {Ò}{{\`O}}1 {Ù}{{\`U}}1
  {ä}{{\"a}}1 {ë}{{\"e}}1 {ï}{{\"i}}1 {ö}{{\"o}}1 {ü}{{\"u}}1
  {Ä}{{\"A}}1 {Ë}{{\"E}}1 {Ï}{{\"I}}1 {Ö}{{\"O}}1 {Ü}{{\"U}}1
  {â}{{\^a}}1 {ê}{{\^e}}1 {î}{{\^i}}1 {ô}{{\^o}}1 {û}{{\^u}}1
  {Â}{{\^A}}1 {Ê}{{\^E}}1 {Î}{{\^I}}1 {Ô}{{\^O}}1 {Û}{{\^U}}1
  {œ}{{\oe}}1 {Œ}{{\OE}}1 {æ}{{\ae}}1 {Æ}{{\AE}}1 {ß}{{\ss}}1
  {ç}{{\c c}}1 {Ç}{{\c C}}1 {ø}{{\o}}1 {å}{{\r a}}1 {Å}{{\r A}}1
  {€}{{\EUR}}1 {£}{{\pounds}}1,
}
\lstset{style=c++}


\title{Simulation TP5\\Système Multi-Agents}
\author{Arquillière Mathieu - Zangla Jérémy}
\date{\today}

\begin{document}

\begin{titlepage}
  \maketitle
\end{titlepage}

\tableofcontents
\newpage
\listoffigures
\newpage

\section{Introduction}
L'objectif ici est de réaliser un système multi-agents simple, afin d'en
comprendre et d'en expérimenter les difficultés et les intérêts.

\section{Modélisation}
\subsection{Agents}
Notre système se composera de 2 types d'agents:
\begin{itemize}
  \item les \emph{récolteurs}
  \item les \emph{mangeurs}
\end{itemize}
Il y a également dans l'environnement
\begin{itemize}
  \item les ressources
  \item les bases (concernants les récolteurs)
\end{itemize}
Une ressources a juste une position dans l'espace 2D et une quantité.

\subsubsection{Récolteurs}
L'objectif de cet agent est récolter des ressources et de les ramener à sa base. Chaque
agent de ce type a été créé par une \emph{case particulière} de l'environnment qu'on
appelera sa \emph{base}. Les \emph{récolteurs} bougent aléatoirement dans l'espace
en 2 dimensions. Lorqu'ils trouvent des ressources, ils la prennent et se déplacent alors
vers leur base (ce qui implique qu'ils savent à tous moments l'endroit de leur base).
Une fois dessus, ils déposent leurs ressources et se remettent en recherche.

Les \emph{récolteurs} naissent grâce à une base. Lorsqu'une base possède assez de ressources,
elle les consomme et créé un nouvel agent \emph{récolteur}.

\subsection{Environnement}
L'environnement sera une matrice (20x20 à la base). Une case de cette matrice peut:
\begin{itemize}
  \item être vide
  \item contenir un agent
  \item contenir une ressource
  \item contenir une base
\end{itemize}

\end{document}