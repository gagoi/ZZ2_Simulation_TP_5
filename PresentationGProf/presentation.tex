\documentclass[a4paper]{article}
\usepackage{makeidx}
\usepackage{formular}
\usepackage{listings}
\usepackage[utf8]{inputenc}
\usepackage[dvipsnames]{xcolor}
\usepackage{mdframed}
\usepackage{multicol}
\usepackage{geometry}

\geometry{hmargin=2.2cm,vmargin=1.5cm}
\definecolor{light-gray}{gray}{0.95} %the shade of grey that stack exchange uses

\title{Présentation GProf}
\author{Jérémy ZANGLA}

\begin{document}

\maketitle
\pagebreak
\tableofcontents
\pagebreak
\section{Présentation}
\subsection{Profiling}
Le \emph{profiling} est le processus qui analyse l'exécution d'un programme.
Il permet d'étudier les performances de notre application, afin de trouver les points à améliorer.
Les domaines que l'on peut étudier sont diverses : temps d'éxécution, espace mémoire, nombre et fréquence d'appel des fonctions.
On appelle \emph{profiler} l'outil qui nous permet de faire du profiling.
\subsection{Gprof}
L'outil \emph{Gprof} est un profiler pour les langages C et C++. 
\end{document}