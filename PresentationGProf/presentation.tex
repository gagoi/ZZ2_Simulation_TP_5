\documentclass[a4paper]{article}
\usepackage{makeidx}
\usepackage{formular}
\usepackage{listings}
\usepackage[utf8]{inputenc}
\usepackage[dvipsnames]{xcolor}
\usepackage{mdframed}
\usepackage{multicol}
\usepackage{hyperref}
\usepackage{geometry}

\geometry{hmargin=2.2cm,vmargin=1.5cm}
\definecolor{light-gray}{gray}{0.95} %the shade of grey that stack exchange uses

\title{Présentation GProf}
\author{Jérémy ZANGLA}

\begin{document}

\maketitle
\pagebreak
\tableofcontents
\pagebreak
\section{Présentation}
\subsection{Profiling}
Le \emph{profiling} est le processus qui analyse l'exécution d'un programme.
Il permet d'étudier les performances de notre application, afin de trouver les points à améliorer.
Les domaines que l'on peut étudier sont diverses : temps d'éxécution, espace mémoire, nombre et fréquence d'appel des fonctions.
On appelle \emph{profiler} l'outil qui nous permet de faire du profiling.
\subsection{Outils}
Il existe différents outils, appelés \emph{profilers}, permettant le profiling en fonction du langage de programmation choisis. On citera par exemple : 
\begin{itemize}
    \item C/C++ : Valgrind (Callgrind, Cachegrind, ...), ...
    \item Java : Java Runtime Analysis Toolkit (JRAT), ...
    \item Python : cProfile, ...
\end{itemize}
Certains de ces outils doivent être intégrés au programme, d'autres sont lancés indépendamment en donnant le programme (les sources dans le cas d'un langage interprété).
\section{GProf}
\subsection{Présentation}Le profiler traité ici est Gprof, il permet d'analyser l'analyse de programmes développés en C ou en C++.
Il appartient à \emph{GNU Bininary Utilities} et est maintenu par le projet GNU.
\subsection{Installation}
Ce logiciel n'est disponnible que sous Linux. La plupart des distributions permettent de l'installer directement depuis les repositories de leur gestionnaire de paquets. 
Cependant on peut l'installer manuellement depuis un linux, en téléchargeant l'archive depuis le (\href{http://ftpmirror.gnu.org/binutils}{"site officiel de GNU"}) et en suivant les instructions du fichier \emph{README}.
\subsection{Utilisation}
\end{document}