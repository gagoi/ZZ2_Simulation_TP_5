\documentclass[a4paper, 12pt]{article}
\usepackage{makeidx}
\usepackage{formular}
\usepackage{listings}
\usepackage[utf8]{inputenc}
\usepackage[dvipsnames]{xcolor}

\usepackage{mdframed}
\usepackage{multicol}
\usepackage{hyperref}
\usepackage{geometry}
\usepackage{mdframed}
\usepackage{listings}

\geometry{hmargin=2.2cm,vmargin=1.5cm}
\definecolor{light-gray}{gray}{0.95} %the shade of grey that stack exchange uses

\title{Présentation g++}
\author{Mathieu ARQUILLIERE}

\lstset{
  literate=
  {á}{{\'a}}1 {é}{{\'e}}1 {í}{{\'i}}1 {ó}{{\'o}}1 {ú}{{\'u}}1
  {Á}{{\'A}}1 {É}{{\'E}}1 {Í}{{\'I}}1 {Ó}{{\'O}}1 {Ú}{{\'U}}1
  {à}{{\`a}}1 {è}{{\`e}}1 {ì}{{\`i}}1 {ò}{{\`o}}1 {ù}{{\`u}}1
  {À}{{\`A}}1 {È}{{\'E}}1 {Ì}{{\`I}}1 {Ò}{{\`O}}1 {Ù}{{\`U}}1
  {ä}{{\"a}}1 {ë}{{\"e}}1 {ï}{{\"i}}1 {ö}{{\"o}}1 {ü}{{\"u}}1
  {Ä}{{\"A}}1 {Ë}{{\"E}}1 {Ï}{{\"I}}1 {Ö}{{\"O}}1 {Ü}{{\"U}}1
  {â}{{\^a}}1 {ê}{{\^e}}1 {î}{{\^i}}1 {ô}{{\^o}}1 {û}{{\^u}}1
  {Â}{{\^A}}1 {Ê}{{\^E}}1 {Î}{{\^I}}1 {Ô}{{\^O}}1 {Û}{{\^U}}1
  {œ}{{\oe}}1 {Œ}{{\OE}}1 {æ}{{\ae}}1 {Æ}{{\AE}}1 {ß}{{\ss}}1
  {ű}{{\H{u}}}1 {Ű}{{\H{U}}}1 {ő}{{\H{o}}}1 {Ő}{{\H{O}}}1
  {ç}{{\c c}}1 {Ç}{{\c C}}1 {ø}{{\o}}1 {å}{{\r a}}1 {Å}{{\r A}}1
  {€}{{\EUR}}1 {£}{{\pounds}}1,
  numbers=left,
  numbersep=10pt,
  showspaces=false,
  showstringspaces=false,
  showtabs=false,
  stepnumber=1,
  stringstyle=\color{gray},
  tabsize=4,
  basicstyle=\small,
  keywordstyle=\bf\color{blue},
  backgroundcolor=\color{light-gray},
  commentstyle=\color{ForestGreen},
  showstringspaces=false
}


\begin{document}

\maketitle
\pagebreak
\tableofcontents
\pagebreak
\section{Présentation}
    La \emph{compilation} est l'ensemble des étapes qui permettent de convertir le code source d'un programme, en un fichier compréhensible par une machine.
    Il y a plusieurs étapes lors de la compilation. La première est l'étape de préprocesseur, on vient remplacer les directives (les lignes commençant par \emph{\#} en C et C++).
    Ensuite le compilateur vérifie que le code source est valide, si c'est le cas il procédera alors à l'étape de compilation des sources en fichiers objets. Enfin il restera à rassembler les fichiers pour ne former qu'un exécutable (assemblage).

    Nous allons ici étudier le compilateur linux g++ qui se charge de compiler des sources provenant du C++.
    Ce compilateur est disponnible sous Linux, et est le compilateur par défaut dans beaucoup de situation.

\section{Utilisation}
    La commande basique pour compiler avec cet outils est la suivante :
    \begin{mdframed}[backgroundcolor=light-gray, roundcorner=20pt,
        innerleftmargin=20, innertopmargin=1, innerbottommargin=1, 
        outerlinewidth=1, linecolor=darkgray]
        \begin{lstlisting}[language=Bash]
g++ source.cpp
        \end{lstlisting}
    \end{mdframed} 
\end{document}